\documentclass{article}
\usepackage[utf8]{inputenc}
\usepackage{todonotes}

\newcommand{\tristan}[1]{\color{orange}\textbf{From Tristan:}#1\color{black}}


\title{PyTracer: Detecting numerical instabilities in Python using Monte Carlo Arithmetic \tristan{title sounds like PyTracer is an MCA tool, emphasis should be put on comparing MCA traces. ``without instrumentation" or ``automatic" could also appear in the title.}}
\author{}
\date{November 2020}

\usepackage{natbib}
\usepackage{graphicx}

\newcommand{\Yohan}[1]{
\todo[inline,backgroundcolor=green]{YC: #1}
}

\begin{document}

\maketitle

\section{Introduction}
\begin{itemize}
    \item Numerical instabilities are inherent to HPC applications \tristan{not only to HPC, also software updates, small perturbations in parameters or data}
    \item Neuroimaging tools are facing reproducibility issues, one factor of which is numerical stability \tristan{neuroimaging should be an illustration, your intro should rather focus on Python in general, it's much more impactful} 
    \item Detecting those instabilities is challenging: large data sets, large pipeline, long computation time, complex physics \tristan{Here you could illustrate the point with neuroimaging pipelines}
    \item Needs tools for automatically detecting these instabilities
    \item Needs tools to quickly identify unstable functions among large number of calls
\end{itemize}

\subsection{Contributions}
\begin{itemize}
    \item Pytracer, a tool to automatically instrument Python functions 
    to trace their inputs and outputs fidelity over time
    \item A visualizer that allows visualizing large traces 
    \item A graph-based summary view to easily identify numerically unstable functions 
    \item \tristan{An evaluation of the above tools on Scipy libraries}
\end{itemize}

\section{Motivating example}
\Yohan{Find motivating example}
\tristan{Could be Greg's pipeline and paper}

\section{Background}
\subsection{Monte Carlo Arithmetic}
\begin{itemize}
    \item Simulation of roundoff errors and cancellations
    \item Virtual precision to simulate any working precision
    \item A formula to compute the number of significant digits, extended with Confidence Interval for Stochastic Arithmetic
\end{itemize}
\subsection{Fuzzy environment}
\begin{itemize}
    \item Python environment using MCA
    \item Transparent for the user, no recompilation needed
    \item Extensible with modules using MCA like numpy
\end{itemize}

\subsection{Tracing tools}
\begin{itemize}
    \item Veritracer
    \item CADNA
    \item CraftHPC
    \item FpDebug
    \item Herbgrind
    \item \Yohan{Missed other tools, read literature}
\end{itemize}

\section{Design principles}

\subsection{Pytracer workflow}

\Yohan{Add figure for the workflow}

\begin{itemize}
    \item Trace application
    \item Merge traces
    \item Visualize trace
\end{itemize}

\subsection{Instrumentation}
Pytracer dynamically instruments the Python module selected by the user 
without intervention required by the user.
Special attributes of the form \texttt{\_\_*\_\_} are not wrapped.

\subsubsection{Functions' module \tristan{Module functions?}}
\begin{itemize}
    \item Iterates over Python's modules selected by the user.
    \item Dynamically compiles a wrapper for each function thanks to \texttt{compile} builtin function.
    \item Replace original function by its wrapper counterpart.
\end{itemize}
\subsubsection{Class instances}
    Some class forbid overwriting attributes of their instances \tristan{why do you need to edit attributes and not just functions?}.
The solution consisting in replacing function by a wrapper cannot be applied here.
\begin{itemize}
    \item Class is not a base type, i.e. can be inherited, we create a new class from the original class by filling its namespace with the callable attributes being replaced by their wrapper counterpart.
    \item Class is a base type, as for \texttt{numpy ufunc}, no 
    general solution can be provided since heritage is forbidden.
    Solutions on a case-by-case basis are provided. 
    \item \texttt{numpy.frompyfunc} can cause errors since signature is not preserved. The function wrapped with this technique always returns \texttt{Python Object} that cause \texttt{TypeError} when the numpy array
    over witch the function is applied is used for indexing another array.
    Tracing \texttt{ufunc} instances can be disabled by the user.
\end{itemize}


\subsection{Visualization}

\subsubsection{Plotly}


\section{Methodology \tristan{not the right title: design is part of the methodology. I feel that "running applications with fuzzy environment" will overlap quite a  bit with 3.2. You could just move "detecting instabilities" to section 4 and forget about this section.}}
\subsection{Running application within Fuzzy environment}
\subsection{Detecting instabilities}


\section{Use cases}

\subsection{Numpy}
\subsection{Scikit learn}
\subsection{Dipy}

\section{Performance evaluation}
\subsection{Overhead}
\subsection{Postprocessing}

\section{Discussion}

\section{Conclusion}

\bibliographystyle{plain}
\bibliography{references}
\end{document}
